\section{Introduction}

The aim with this thesis was to genotype the 270 base pair long second exon of the gene DLA-DRB1, located in the in {M}ajor {H}istocompatibility {C}omplex ({MHC}), in 3200 dogs. The idea was to use the results to find out in which dogs we can see the biggest genetic diversity, information that can be used in the investigation of where in the world that dogs originate.

The {MHC} is a hypervariable region of the genome in all vertebrates that contains genes that play a major role in the immune system. For humans MHC is often referred to as Human Leukocyte Antigen (HLA) and in dogs as Dog Leukocyte Antigen (DLA). This region code for proteins situated on the cell surface that helps to distinguish between self and non-self. {MHC} is of interest for clinical science since changes can have effect on disease defense, cancer and autoimmune disease, which also make it important for organ transplantation \cite{hla_typing}. The MHC class II region is the most variable protein coding region known, this make it an excelent region to study in population genetics. The HLA has been used to reconstruct human migration events \cite{abdennaji06,di11,buhler06}. The extreme variability makes genotyping the MHC difficult, it is a well known problem in the world of bioinformatics.\\
The original data set consisted of 4708 dog- and wolf specimens from all continents around the world. These animals were sampled from hair (819), blood(1830) and FTA cards (2059), which is a technique to collect cells from the saliva of an individual \cite{neiman11}.\\
We will show a Needleman-Wunsh\cite{nw} based method to align sequences from a hypervariable region that contains artefacts in the form of read errors, machine induced insertions and deletions (indels) and chimeric formation. Moreover we will propose a technique to find the true alleles in a noisy dataset like this. 
We will go through background theory to get some understanding of how and why this study was made. After this we'll have a closer look at the actual problems and show the method that we used to handle them. We finish of with a review of the results and a discussion of what can be done to improve the results.\\

